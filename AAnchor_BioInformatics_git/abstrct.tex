
The recent cryo-EM resolution revolution enables the development of algorithms for  direct de-novo modeling of protein structures into cryo-EM density maps. 
Such anchor residues can be exploited in several local de-novo modeling tasks, such as the reliable positioning of secondary
structures, loop modeling and general fragment based modeling.
\\
\textbf{Results:} 
A deep learning based method was developed for the detection of  high confidence anchor amino acid residues in such a map. 
In the experimental results we show the ability of the proposed procedure to locate and classify a significant number of amino acids in  density maps of $3.1$ {\AA}  (or better) resolution. 
Our performance analysis indicates that the main factor affecting the detection accuracy is the  lack of sufficient experimental data for the training stage of the algorithm.
Thus, our method is expected to improve significantly in the near future, due to the  rapid increase in the release of novel high resolution cryo-EM maps.
\\
\textbf{Availability:} A web application based on the method can be found at \url{https://bioinfo3d.cs.tau.ac.il/AAnchor}\\
\textbf{Contact:} {markroza@tauex.tau.ac.il}\\
\textbf{Supplementary information:} Supplementary data are available at \textit{Bioinformatics}
online.}