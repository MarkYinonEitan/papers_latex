In this paper we present a CNN based method for localization and classification of amino acids in high resolution cryo EM maps.
Whilst the \textit{de-novo} detection of all protein residues is still a hard problem, we succeeded to detect with high confidence a significant percentage of some amino acids.  
Experimental results show that the proposed method is capable of detecting a sufficient number of amino acid "anchors" in a cryo Em map of resolution $3.1 \AA$ or higher.
The reported confidence of a detection is at least as the ground truth accuracy.  These anchors can be further exploited in conjunction with several proposed modeling techniques as well as in the development of novel modeling methods.

We analyzed the detection process and factors affecting the classification task and concluded
that the number of rotamers for a given residue and the size of the training data set have a dominant effect on  the  detection rate, while the amino acid size plays a secondary role.
Contrary to expectation, the results for $2.9-3.1 \;\AA$ resolution maps have a  better detection potential than the more accurate $2.2 \;\AA$ maps. This is due to the limited training data set existing for the better resolution samples.
Also the detection accuracy  can be significantly improved by combining simulated and experimental cryo EM maps to compensate for the lack of experimental data.

As the number of released high resolution maps grows \cite[]{Lawson2016}, the detection accuracy will rise. 
Applying post processing techniques such as Linear Regression \cite[]{Naseem2010} and SVM \cite[]{Girshick2014} may potentially decrease the number of false detections.
Hopefully, with the accumulation of a sufficiently large number of experimental maps the classification of all voxels in a map to the different amino acids will be solvable using the Fully Convolutional CNN \cite[]{Long2015} technique.
