\section{Abstract}
Cellular processes are performed and regulated by assemblies of macromolecules. 
Full understanding of the function, associations and dynamics of  such
assemblies comes from their detailed atomic structural description. 
Traditionally structural analysis of a complex is made by integration of the results of various experimental techniques: X-ray crystallography, NMR, cryo electron microscopy (cryo-EM)  and others.
The output of a cryo-EM experiment is a 3D density map.
In recent years,  vast progress was achieved in obtaining high-resolution (3.5\AA~ and below) cryo-EM maps.
However,  modelling an atomic structure from a cryo EM  map remains difficult.
For high-resolution maps  researchers mostly rely on
methods developed for X-ray crystallography. 
Modelling protein assemblies into medium resolution maps is usually based on image processing techniques and has still not yet achieved good performance.

Parallel to the advances in Cryo-EM in the last decade, deep neural networks achieved remarkable performance in various 2D and 3D image processing tasks. 
Those include Convolutional Neural Networks (CNNs) for image classification, Fully Convolutional Networks (FCN) for semantic image segmentation, Variational AutoEncoders (VAEs) and Generative Adversarial Networks (GANs) for realistic image generation.

We propose a Deep Learning Analysis Framework (DLAF) for integrative structural analysis of cryo-EM maps.
The proposed framework integrates 3D imaging information from a cryo-EM map with existing sequence and structural data.
Deep Convolutional Networks  are used to locate structural motifs within a map.
CNNs are trained on a dataset adjusted to a specific problem.
This adjustment is  done using the sequence and structural data.
Realistic cryo-EM map simulation plays a key role in the presented approach.
While the simulation of experimental maps is crucial for DLAF, it is also a valuable tool for development and analysis of algorithms which work with cryo-EM and can contribute a lot to the structural bioinformatics community.
Preliminary results of our study show that CNN are capable to successfully detect amino acids in high-resolution cryo-EM maps.
We also developed a realistic simulation of an experimental cryo-EM map using an adversarial deep learning approach.